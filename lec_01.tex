% !TEX root = /home/computer/ucsc/master-2/quarter-1/computational-fluids/master.tex
\lecture{1}{Mon 04 Oct 2021 11:32}{Mass Conservation}
\subsectionfont{\fontsize{10}{10}\selectfont}

\section{Substational Derivative}%

Let $\vec{u} = \vec{u}(x,y,z,t) = <u(x,y,z,t), v(x,y,z,t), w(x,y,z,t)>$. Then the total derivative

\[
  d\vec{u} = \diffp[]{\vec{u}}{x}dx + \diffp[]{\vec{u}}{y}dy
  + \diffp[]{\vec{u}}{z}dz + \diffp[]{\vec{u}}{t}dt
.\] 

\begin{figure}[ht]
    \centering
    \incfig{total-derivative}
    \caption{Total derivative}
    \label{fig:total-derivative}
\end{figure}


\begin{align*}
  \diff[]{\vec{u}}{t} &= \diffp[]{\vec{u}}{x}\diff[]{x}{t}
  + \diffp[]{\vec{u}}{y}\diff[]{y}{t} + \diffp[]{\vec{u}}{z}\diff[]{z}{t}
  + \diffp[]{\vec{u}}{t} \\
                      &= \diffp[]{\vec{u}}{x}u + \diffp[]{\vec{u}}{y}v
  + \diffp[]{\vec{u}}{z}w + \diffp[]{\vec{u}}{t}
\end{align*}

Lagrangian view of a fluid element (substantial derivative) measures the time
rate of change of a given quantity (fluid velocity in this case) as it
moves from one location to another in both space and time.
\[
  \boxed{\frac{D \vec{u}}{Dt} :=
    \underbrace{(\vec{u}\cdot\nabla)\vec{u}}_\text{advection term}
  + \underbrace{\diffp[]{\vec{u}}{t}}_\text{Eulerian view}}
.\] 

As an operator,
\begin{equation} \label{eq:SD}
  \boxed{\frac{D}{Dt} = \frac{\partial}{\partial t} + (\vec{u}\cdot\nabla)}
\end{equation}


\section{Mass Conservation - continuity equation}%
\subsection{Finite Control Volume (FCV) $V$ fixed in space with fluid moving through
it}%

\begin{figure}[ht]
    \centering
    \incfig{fcv-fixed-in-space}
    \caption{FCV fixed in space}
    \label{fig:fcv-fixed-in-space}
\end{figure}

Total mass in volume
\[
  \textcolor{teal}{M_{v}} = \int_{V} { \rho} \: d{V}
.\] 

mass flux through volume
\[
  \int_{S} { \rho \vec{u} \cdot \hat{n}} {}  \: {dS} {}
.\]

By conservation, change in mass in volume equals mass flux out
\begin{align*}
  \frac{\partial M_{v}}{\partial t} &= - \int_{S} { \rho \vec{u} \cdot \hat{n}} {}  \: {dS} {} \\
  \frac{\partial }{\partial t} \int_{V} { \rho} \: d{V} &= - \int_{S} { \rho \vec{u} \cdot \hat{n}} {}  \: {dS} {} \\
\end{align*}

This results in the conservative form of the continuity equation in integral
form.
\begin{equation} \label{eq:F1}
  \boxed{\frac{\partial}{\partial t} \int_{V} { \rho} \: d{V} + \int_{S} { \rho \vec{u} \cdot \hat{n}} {}  \: {dS} {} = 0} \\
\end{equation}


\subsection{Finite Control Volume $V$ moving with the fluid}%

\begin{figure}[ht]
    \centering
    \incfig{fcv-moving-with-the-fluid}
    \caption{FCV moving with the fluid}
    \label{fig:fcv-moving-with-the-fluid}
\end{figure}

Taking the substantial derivative of the mass $M_{v}$ of a moving control
volume. The change in mass of this control volume is zero as it moves
with the fluid.
\begin{equation} \label{eq:F2}
  \frac{DM_{v}}{Dt} = \boxed{\frac{D}{Dt}\int_{V} { \rho} \: d{V} = 0}
\end{equation}

This is the non conservative form of the continuity equation in integral form.

\subsection{Infinitesimal Fluid Element (IFE) $dV$ fixed in space with fluid moving
through it}%

\begin{figure}[ht]
    \centering
    \incfig{ife-fixed-in-space}
    \caption{IFE fixed in space}
    \label{fig:ife-fixed-in-space}
\end{figure}

Conservative form in differential form.

\subsection{Infinitesimal Fluid Element $dV$ moving along a streamline}%
Non conservative form in differential form.

\begin{figure}[ht]
    \centering
    \incfig{ife-moving-along-streamline}
    \caption{IFE moving along streamline}
    \label{fig:ife-moving-along-streamline}
\end{figure}


\section{Bringing everything together}%

From equation \ref{eq:F1}, if we assume the volume doesn't change in time, 
we can bring the derivative inside the integral
\[
   \int_{V} { \frac{\partial\rho}{\partial t}} \: d{V} + \int_{S} { \rho \vec{u} \cdot \hat{n}} {}  \: {dS} {} = 0 \\
.\] 

Then, applying the divergence theorem, we can write the surface integral as
a volume integral and combine terms

\begin{align*}
  \int_{V} { \frac{\partial\rho}{\partial t}} \: d{V} + \int_{V} { \nabla\cdot\rho \vec{u}
  } \: {dV} &= 0 \\
\end{align*}

We get an alternative form of FCV fixed in space (F1)

\begin{equation}
  \boxed{\int_{V} { \frac{\partial\rho}{\partial t} +  \nabla\cdot\rho \vec{u}}
  \: d{V} = 0}
\end{equation}

Since this is an arbitrary volume, this implies IFE fixed in space (F3)

\begin{equation}
  \boxed{ \frac{\partial\rho}{\partial t} +  \nabla\cdot\rho \vec{u} = 0}
\end{equation}

Then applying the vector identity $\nabla\cdot\rho \vec{u} = \vec{u}\cdot\nabla \rho + \rho\nabla\cdot \vec{u}$
\[
  \int_{V} { \frac{\partial\rho}{\partial t} + \vec{u}\cdot\nabla \rho + \rho\nabla\cdot \vec{u}} \: d{V} = 0
.\] 

Noting the use of the substantial derivative, we get FCV moving in space (F2)

\begin{equation}
  \boxed{\int_{V}{\frac{D \rho}{Dt} + \rho\nabla\cdot \vec{u}} \: d{V} = 0}
\end{equation}

Since this is an arbitrary volume, we get IFE moving in space (F4)
\begin{equation}
  \boxed{\frac{D \rho}{Dt} + \rho\nabla\cdot \vec{u} = 0}
\end{equation}

So (F1) $\implies$ (F3) and (F1) $\implies$ (F2) $\implies$ (F4)
