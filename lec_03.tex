% !TEX root = /home/computer/ucsc/master-2/quarter-1/computational-fluids/master.tex
\lecture{3}{Thu 04 Nov 2021 09:31}{Euler Equation}

\subsectionfont{\fontsize{10}{10}\selectfont}

\section{Euler Equation}%

Conservation of momentum (Force (net) $=$ rate of change of momentum of
lagrangian volume changing with time)
\begin{equation}
  \frac{D \vec{u}}{Dt} = - \frac{1}{ \rho}\vec{\nabla}p + g 
\end{equation}

Using substantial derivative
\begin{equation}
  \underbrace{\diffp[]{\vec{u}}{t}}_\text{eul acc at
    a pnt} + \underbrace{( \vec{u}\cdot \vec{\nabla}
    ) \vec{u}}_\text{advection lag acc} = \underbrace{
  \underbrace{-\frac{1}{\rho}\vec{\nabla}p}_\text{stresses press grad} + \underbrace{g}_\text{body/grav/boyncy}}_\text{driving forces}
\end{equation}


\section{One-dimensional time-dependent Euler Equation}%
\subsection{Euler equation in compact conservative form. }%

\begin{align}
  U_{t} + \nabla\cdot F(U) &= 0 \\
  U_{t} + F(U)_{x} &= 0 \qquad\qquad\quad\nabla = \frac{\partial}{\partial x}\\
  U_{t} + A(U)U_{x} &= 0
\end{align}

where $A(U) = \frac{\partial{F(U)}}{\partial{U}}$ is the flux Jacobian matrix.
Goal diagonalize $A(U) = R\Lambda R^{-1}$, then the Euler Equations, for $W := R^{-1}U$, can be
written as,

\begin{align}
  R^{-1}U_{t} + \Lambda R^{-1}U_{x} &= 0 \\
  W_{t} + \Lambda W_{x} &= 0 \\
  \diffp[]{w_{k}}{t} + \lambda_{k} \diffp[]{w_{k}}{x} &= 0, \qquad k=1,2,\dots,m
\end{align}

and becomes completely decoupled family of individual scalar equations with solution $\boxed{U = RW}$

\subsection{The Conservative-Variable Form of Euler Equations}%

\begin{equation}
  U = \begin{pmatrix}
    \rho \\ \rho u \\ E
  \end{pmatrix}
\end{equation}

where $E$ is total energy per unit volume,

\[
E = \rho \left( \frac{u^{2}}{2} + e \right)
.\] 

and $e$ is the specific internal energy given by caloric Equation of State
(EoS)

\[
e = e( \rho, P)
.\] 

For ideal gases

\[
  e = \frac{p}{ ( \gamma - 1) \rho}
.\] 

\subparagraph{The vector of conservative fluxes is given by}%

\[
  F(U) = \begin{pmatrix}
    \rho u \\ \rho u^{2} + p \\ u(E+p)
  \end{pmatrix}
.\] 

\paragraph{Diagonalization given on page $130$}%
\[
  L^{c}A(U)R^{c} = \Lambda
.\] 


\subsection{The Primitive-Variable Form of the Euler Equations}%

\begin{equation}
  V = \begin{pmatrix}
    \rho \\ u \\ p
  \end{pmatrix}
\end{equation}

\paragraph{Diagonalization given on page $131$}%
\[
  L^{p}A(V)L^{p} = \Lambda
.\] 

\paragraph{Note}%
$A(V)$ and  $A(U)$ are similar, for $Q = \diff[]{U}{V}$ and $Q^{-1}
= \diff[]{V}{U}$,

\[
  A(V) = Q^{-1}A(U)Q
.\] 

and hence have the same eigenvalues.



