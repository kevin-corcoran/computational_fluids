% !TEX root = /home/computer/ucsc/master-2/quarter-1/computational-fluids/master.tex
\lecture{2}{Thu 14 Oct 2021 05:13}{Burgers' Equation}

\subsectionfont{\fontsize{10}{10}\selectfont}

\subsection{Linear scalar equations}%
\paragraph{The advection equation with constant velocity $a$}
\subparagraph{In non-conservative form}
\[
u_{t}+au_{x} = 0
.\] 

with initial condition
\[
  u(x,0) = u_{0}(x)
.\] 

Considering a parameterization of the solution $u(x,t)$
 \[
   u(x(s),t(s))
.\] 

Taking the derivative of $u$ with respect to $s$ 

\[
  \diff[]{u}{s} = \diff[]{t}{s}u_{t} + \diff[]{x}{s}u_{x}
.\] 

Yields the characteristic equations
\[
\begin{dcases}
  \diff[]{t}{s} = 1 \\
  \diff[]{x}{s} = a \\
  \diff[]{u}{s} = 0 
\end{dcases}
.\] 

with solutions
\[
\begin{dcases}
  t(s) = s + t_0 \\
  x(s) = as + x_0 \\
  u(s) = u_0 = u(x_0,t_0) = u(x_{0},0) = u_{0}(x_{0})
\end{dcases}
.\] 

since $t_0=0$ then $s=t$ and $x_0 = x - at$, so 
\[
  \boxed{u(x,t) = u_0(x-at)}
.\] 

Where $x-at$ is the characteristic line with a given $x_0$ and propagation
velocity $a$. So the initial data $u_0(x)$ is advected to the right for $a>0$
and to the left for $a<0$

\begin{figure}[ht]
    \centering
    \incfig{linear-advection}
    \caption{linear advection}
    \label{fig:linear-advection}
\end{figure}


\paragraph{The advection equation with variable velocity $a(x(t))$}

\subparagraph{In conservative form}
\[
  u_{t}+ \left(a(x(t))\right)_{x} = 0
.\] 

\subparagraph{In non-conservative form}
\[
  u_{t}+a(x(t))u_{x} = -a'(x(t))u
.\] 

and we can obtain the characteristic equations
\[
  \begin{dcases}
  \diff[]{u(x(t))}{t} = -a'(x(t))u \neq 0\\
  \diff[]{x(t)}{t} = a(x(t)) + x_0
  \end{dcases}
.\] 

so the solution is no longer constant along characteristics. Further, the
characteristics are no longer straight lines.

\subsection{Domain of Influence}%

\subsection{Range of Influence}%

\subsection{Nonlinear scalar equations}%

\subsection{Greens functions}%


\paragraph{In conservative form}
\[
  u_{t}+ \left(f(u)\right)_{x} = 0
.\] 

where $f(u)$ is a nonlinear function of $u$ and is called the flux function.
There are two types of flux functions

\begin{enumerate}
  \item $f(u)$ convex - i.e., $f''(u)>0 \forall u$. (of concanve)
  \item $f(u)$ is non-convex
\end{enumerate}


\paragraph{In non-conservative form}
\[
  u_{t} + \diff[]{f(u)}{u}f_{x}(u) = 0
.\] 

where $\diff[]{f(u)}{u}$ is called the characteristic speed

\subsection{Burgers' Equation}%

Solutions for $t < t_{s}$ and for $t > t_{s}$ $\dots$

\subsection{Weak solutions}%

$\bf{Definition:}$ The function $u(x,t)$ is called a $\emph{weak solution}$ of
the scalar conservation law $u_{t}+(f(u))_{x} = 0$ if it satisfies the
following for all test functions $\phi(x,t)\in C_{0}^{1}(\R \times \R^{+})$:

\subsection{Example}%
Want to solve
\[
\begin{dcases}
  u_{t}+ \left( \frac{u}{2} \right)_{x} = 0 \\
  u(x,0) = g(x) = 
    \begin{cases}
      1, & x < -1 \\
      -x, & -1 \leq x \leq 0 \\
      0, & x > 0
    \end{cases}
\end{dcases}
.\] 

\begin{figure}[ht]
    \centering
    \incfig{example-initial-condition-for-burgers}
    \caption{example initial condition for burgers}
    \label{fig:example-initial-condition-for-burgers}
\end{figure}

Assuming $u(x,t)$ is smooth. In conservative form
\[
  u_{t}+ \left( \frac{u^{2}}{2} \right)_{x} = u_{t}+uu_{x} = 0
.\] 

Considering the parameterization of the solution $u(x,t)$
 \[
   u(x(s),t(s))
.\] 

and then taking the derivative of $u$ with respect to $s$ 
\[
\diff[]{u}{s}=\diff[]{t}{s}u_{t}+\diff[]{x}{s}u_{x}
.\] 

Yields the characteristic equations
\[
\begin{dcases}
  \diff[]{t}{s} = 1 \\
  \diff[]{x}{s} = u \\
  \diff[]{u}{s} = 0
\end{dcases}
.\] 

with solutions
\[
\begin{dcases}
  t(s) = s + t_0 \quad t_0 = 0 \\
  u(s) = u_0 = u(x_0,t_0) = u(x_0,0) = g(x_0) \\
  \implies x(s) = g(x_0)s + x_0 \\
\end{dcases}
.\] 

\begin{figure}[ht]
    \centering
    \incfig{characteristic-lines-example1}
    \caption{characteristic lines example1}
    \label{fig:characteristic-lines-example1}
\end{figure}

This line has slope $g(x_0)$. Since the value of the solution depends on the initial data $x_0$, and the characteristic line depends also on the initial data $x_0$, we'll have to break this up into cases.

\textbf{Case 1.} $x_0<-1$ then $g(x_0)=1$ by definition. Hence
\begin{gather*}
  x(t) = t+x_0 \\
  \boxed{u(x(t),t) = 1}
\end{gather*}

\textbf{Case 2.} $-1\leq x_0\leq 0$ then $g(x_0)=-x_0$ by definition. Hence
\begin{gather*}
  x(t) = -x_0t+x_0\\
  u(x(t),t) = -x_0
\end{gather*}

Solving for $x_0$ we find $x_0= \frac{x(t)}{1-t}$ and
\[
  \boxed{u(x(t),t) = \frac{x(t)}{t-1}}
.\] 

\textbf{Case 3.} $x_0>0$ then $g(x_0)= 0$ and
\begin{gather*}
  x = x_0 \\
  \boxed{u(x(t),t) = 0}
\end{gather*}

so the final solution is
\[
  \boxed{u(x(t),t) = 
        \begin{cases}
            1, & x < t-1 \\
            - \frac{x(t)}{1-t}, & t-1\leq x\leq 0\\
            0, & x>0
        \end{cases}}
.\] 

for $t<1$ as $t$ approaches $1$ we get a discontinuity

\begin{figure}[ht]
    \centering
    \incfig{example1-solution}
    \caption{example1 solution}
    \label{fig:example1-solution}
\end{figure}

Plotting the characteristic lines

\begin{figure}[ht]
    \centering
    \incfig{example1-characteristics}
    \caption{example1 characteristics}
    \label{fig:example1-characteristics}
\end{figure}
